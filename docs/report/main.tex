\documentclass{article}
\usepackage[utf8]{inputenc}
\usepackage[spanish]{babel}
\usepackage{amsthm}
\usepackage{amsmath}
\usepackage{amssymb}
\usepackage{graphicx}
\usepackage{wrapfig}
\usepackage[letterpaper, top=0.78in, bottom=0.78in, left=0.98in, right=0.98in]{geometry}
\usepackage[hidelinks]{hyperref}
\usepackage{url}
\usepackage{soul}

\decimalpoint
\renewcommand{\baselinestretch}{1.5}

\begin{document}
    \begin{titlepage}
        \begin{center}
            % School logo
            \begin{figure}
                \centering
                \includegraphics[scale=0.13]{/home/mona/Pictures/logo_itesm.png}\\ % Logo de la institución
            \end{figure}
            \vspace{5cm}
            % School data
            \LARGE{Instituto Tecnológico y de Estudios Superiores de Monterrey}\\
            \vspace{1cm}
            \large Escuela de Ingeniería y Ciencias \\
            \vspace{0.2cm}
            \large Ingeniería en Ciencias de Datos y Matemáticas \\
            \vspace{0.2cm}
            \large Uso de álgebras modernas para seguridad y criptografía \\
            \vspace{1cm}
            \textbf{Implementación de criptografía de clave pública para protección de comunicaciones con IoT en entornos de monitoreo y consumo de energía.}\\ % Nombre de la tarea
            \vspace{0.7cm}
            % Tabla de integrantes del equipo
            \begin{table}[h!]
                \centering
                \begin{tabular}{ ||c|c|| }
                    \hline
                    Nombre & Matrícula \\
                    \hline
                    Juan Pablo Echeagaray González & A00830646 \\
                    \hline
                    Ricardo Camacho Castillo & A01654132 \\
                    \hline
                    Michelle Yareni Morales Ramón & A01552627 \\
                    \hline
                    Emily Rebeca Méndez Cruz & A00830768 \\
                    \hline
                    Daniela García Coindreau & A00830236 \\
                    \hline
                    Carolina Longoria Lozano & A01721279 \\
                    \hline
                \end{tabular}
            \end{table}
            \vspace{0.7cm}
            \large 	Dr. Alberto F. Martínez \\ % Nombre del profesor 1
            \vspace{0.2cm}
            \large 	Dr. Daniel Otero Fadul\\ % Nombre del profesor 2
            \vspace{0.2cm}
            \large Socio Formador: COCOA, LICORE \\
            \vspace{0.2cm}
            \large Monterrey, Nuevo León \\
            \vspace{0.2cm}
            \large 17 de marzo del 2023 \\
            \vspace{1cm}
        \end{center}
    \end{titlepage}
    % \maketitle

    \tableofcontents
    \listoffigures
    \listoftables
    \clearpage
    \renewcommand{\tablename}{Tabla}

    \section{Introducción}

        \begin{itemize}
            \item Descripción general del trabajo que se realizará, mencionando brevemente el caso de negocio que buscamos resolver
            \item Describir la estructura del reporte técnico
        \end{itemize}

    \section{Objeto de estudio}

        Implementación de (protocolo?) criptográfico para el envío de mediciones de generación y consumo de un panel solar a un centro de control por medio de una conexión WiFi. El centro de control para nuestro caso de estudio es una instancia de servidor de EC2 de Amazon, y el conjunto de auditores que procesan y envían esos datos son dispositivos IoT de bajo poder de procesamiento pero costo reducido, suponemos también un uso menor de energía eléctrica

    \section{Planteamiento del problema}

        Descripción detallada de la situación problema como la presentada en canvas, aquí podemos aprovechar para mostrar diagramas y delimitar el objeto de estudio

    \section{Justificación}

        Justificar por qué nos deberían de dar recursos para el reto

        Descripción de algunos ataques a dispositivos IoT por el mal uso de primitivas criptográficas, creo que no necesita estar específicamente aplicado al caso de estudio de dispositivos de medición de paneles solares, pero si encontramos algún caso, le daría mucha validez al reto, desde aquí le podríamos meter miedo al socio formador

        Justificar los requerimientos base del problema (como los que vienen en canvas de criptografía ligera)

        % Incluir Objetivos de Desarrollo Sostenible en la justificación del proyecto

    \section{Estado del arte}

        Hablar de arquitecturas de red utilizadas en un contexto similar al del reto, ya tenemos algo de P2P, pero tiene que ser algo resumido que después podamos usar para argumentar la selección de red, al final siempre queremos la más sencilla, robusta y económica.

        Hablar también de criptografía de clave pública, necesitamos de esta parte principalmente de casos de estudio en los que se hayan aplicado algunas técnicas como RSA/ECC para la generación de las claves privadas y públicas

        [NOTE] A partir de aquí somos libres de usar las librerías que queramos

        Después debemos de buscar el cómo se hace regularmente el envío de información de un dispositivo IoT hacia un servidor. ¿Hay alguna forma eficiente de realizarlo con alguna librería ya implementada?

        Luego debemos de hacer mención de la parte de CI/CD, qué políticas se establecerán para la actualización del software del auditor? Así como las prácticas criptográficas necesarias para validar el software que se descargue

        Hablar también de algunas técnicas generales de encriptado de bases de datos, en teoría dispondremos de una instancia de EC2 en la que nosotros tendremos que levantar el servidor y hostear la base de datos

    \section{Recursos disponibles}

        Mencionar que tenemos de cota superior 100 dólares, pero que el socio formador prefiere un costo de alrededor de los 50 dólares

        Disponemos de acceso a los servicios de EC2 de Amazon para el despliegue de un servidor que podamos usar para hostear una base de datos

    \section{Objetivos}


    En fines académicos se busca:

        Desarrollo de una primitiva criptográfica de clave pública sin hacer uso de bibliotecas de terceros para el cómputo de las claves. Aclarando que no existen limitaciones para el uso de bibliotecas que realicen de forma eficiente algunas operaciones matemáticas que necesitemos

            En términos ingenieriles:

        Desarrollo de un sistema de monitoreo de consumo y producción eléctrica de una vivienda, dicho sistema enviará cada determinado lapso de tiempo las lecturas pertinentes a un centro de control por medio de una conexión a internet. Se busca también que dicha información se almacene dentro de una base de datos para su análisis futuro.

            Las etapas anteriormente mencionadas

        De forma concreta en términos económicos y de tiempo computacional:

        Se busca que el auditor propuesto tenga un costo inferior a los 100 USD, pero que de preferencia se encuentre en un precio menor a 50 USD

            El envío de las lecturas debe de ocurrir cada 15 minutos

    \section{Arquitectura de red propuesta}

        Hablar más a detalle de la arquitectura propuesta, enunciando qué componente fungirá cada papel, es como lo que el profe dice de "ponerle nombre y apellido", proponer una nomenclatura para generar identificadores únicos?

    \section{Propuesta metodológica}

        Enunciar los pasos del reto en el orden correcto
        [TEMP] Necesito validar con el profe que estos sean los correctos:
        Arranque de la red

    \section{Experimentación y resultados}

        Generar un análisis estadístico sencillo del tiempo de cómputo de nuestra implementación, podemos compararla con el de librerías prestablecidas

        Demostrar que nuestra implementación cumple con una suite de vectores de prueba, que puede estar basada en los de las librerías grandes

    \section{Discusión de resultados}

        En esta parte argumentamos la eficiencia de nuestra implementación con los datos de la sección pasada

    \section{Objetivos de Desarrollo Sostenible}

        Argumentar el por qué se han seguido los 3 ODS propuestos en canvas:

        Objetivo 7. Garantizar el acceso a una energía asequible, segura, sostenible y moderna para todos.

        Objetivo 9. Industria, innovación e infraestructura (ej. Protección de entornos Wireless, Navegación segura por Internet, entornos Near Field Communication (NFC)).

        Objetivo 13. Adoptar medidas urgentes para combatir el cambio climático y sus efectos.

    \section{Medio de contacto}\label{sec:contact}

        El desarrollo del proyecto así como la redacción del presente documento es un trabajo conjunto de:
        \begin{itemize}
            \item Juan Pablo Echeagaray González
            \item Ricardo Camacho Castillo
            \item Michelle Yareni Morales Ramóz
            \item Emily Rebeca Méndez Cruz
            \item Daniela García Coindreaz
            \item Carolina Longoria Lozano
        \end{itemize}

        Así mismo se destacan los siguientes profesores, como asesores y supervisores de los avances en el desarrollo del proyecto:
        \begin{itemize}
            \item Dr. Alberto F. Martínez
            \item Dr. Daniel Otero Fadul
        \end{itemize}

        El benefactor principal del proyecto es la organización \textit{Licore}, la comunicación con la organización se vio llevada principalmente por el Dr. Iván S. Razo-Zapata.

        En caso de encontrar fallas en el código fuente, o que se necesite de una aclaración de la implementación propuesta; se pide que se abra un \textit{issue} en el repositorio en GitHub que puede ser accedido desde la siguiente \href{https://github.com/JuanEcheagaray75/licore-pki}{liga}.

    \clearpage
    \bibliographystyle{IEEEtran}
    \bibliography{references.bib}

\end{document}
